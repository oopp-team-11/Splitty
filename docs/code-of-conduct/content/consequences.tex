\section{The consequences of not keeping agreements}

We decided to have the following consequences for not keeping agreements: They will have to explain themselves to the
rest of the group (including the TA) and apologise.
Examples of not keeping agreements are:
\begin{itemize}
  \item Being more than 30 minutes late or not showing up to a meeting
  \item Not finishing work or not completing a task in time
  \item Not responding to emails / messages or not responding to phone calls within a reasonable time frame
\end{itemize}

If a team member does something that merits a consequence but said action has not been previously discussed,
then we will hold a separate (online) meeting to decide on the consequence we will impose on the offending team member.

Enforcing additional consequences might only lead to animosity among team members,
which is not productive.

Instead of enforcing more consequences, we will focus our effort on helping team members keep agreements.
An example technique we will use is asking team members for follow-up questions about their agreement.
Questions like "how?", "why?" and "when?" force the team members to make an initial commitment by thinking about their agreement.
This acts as a commitment device.
