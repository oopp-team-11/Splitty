\section{Communication}

The primary communication channel should be the Mattermost server. We should keep there most of the discussions about the project and ongoing work. Every debate about essential functionalities and core concepts everyone should understand, as well as important organizational matters, should be held on Mattermost. It is also a place, where we can ask our TA for clarification if we have any questions regarding the course.

We can also communicate through the Discord server, however we should aim to keep discussions there off-topic. Discord may also be used if we want to hold a long conversation that may devolve to spam and is not essential for our understanding of the project or for quick questions, which do not shape the overall state of the project. Gitlab can be used for commenting and reviewing the code of others, and to give or request feedback. If someone wants to change something in a merge request, they should communicate it on Gitlab, providing a sensible explanation. Gitlab issues can also be used as a planning tool - unsolved problems and weekly division of work should be posted as GItlab issues.

If someone cannot participate in a meeting, Discord may also be used for video/audio calls, assuring that hybrid or online meetings are possible.

Discussion regarding any additional meetings should be held openly on Mattermost to ensure that everyone has a chance to participate or suggest schedule adjustments.

Chairman should submit an agenda for the current week before the meeting with TA, so everyone has a chance to read it. During the meeting minute taker should clearly state when they will submit meeting notes, it should be in the evening of the following day the latest.
